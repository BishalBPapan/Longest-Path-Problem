\begin{frame}{Polynomial time variations}
    \centering
    \scalebox{1}{%
    \begin{tabular}{|c|c|c|} \hline
    \textbf{Special graphs} & \textbf{Complexity} & \textbf{Comments} \\ \hline \hline
    Tree & Linear & Dijkstra's algorithm \cite{BULTERMAN200293}\\ \hline 
    Cacti Graph & $O(n^{2})$ & \cite{10.1007/978-3-540-30551-4_74} \\ \hline
    Bipartite Permutation Graph & Linear &  \cite{UEHARA200771}  \\ \hline
    Directed Acyclic Graph & Linear & Dynamic approach \\ \hline
    Interval Graph & $O(n^{4})$ & Dynamic approach \cite{inproceedings} \\ \hline
    Circular Arc Graph & $O(n^{4})$ & Dynamic approach \cite{MERTZIOS2014383} \\ \hline
    Co-compatibility Graph & $O(n^{7})$ & From Hasse diagram \cite{Ioannidou2013} \\ \hline 
    
    \end{tabular}}
    \begin{itemize}
        \item <1> Tree is undirected acyclic graph.Dijkstra proposed an agorithm for this in 1960.
        \item <2>Cacti is a special kind of block graph which each block is a cycle.Two cycle share at most one vertex which is a separator.
    \end{itemize}
\end{frame}
\begin{frame}{Observation}
    \begin{itemize}
        \item The longest path problem is solvable in polynomial time 
on any class of graphs with bounded tree-width or bounded clique-width, 
such as the distance-hereditary graphs.
        \item Finally, it is clearly NP-hard 
on all graph classes on which the Hamiltonian path problem is NP-hard, 
such as on split graphs, circle graphs, and planar graphs.
    \end{itemize}
    
\end{frame}

% \begin{frame}{Polynomial time variations}
%     \centering
%     \scalebox{0.75}{%
%     \begin{tabular}{|c|c|c|} \hline
%     \textbf{Special graphs} & \textbf{Complexity} & \textbf{Comments} \\ \hline \hline
%     Tree & Linear & Dijkstra's algorithm  \citep{BULTERMAN200293}\\ \hline 
%     Bipertite Permutation Graph & Linear &  \citep{UEHARA200771}  \\ \hline
%     Cactus Graph & Linear & \citep{10.1007/978-3-540-30551-4_74} \\ \hline
%     Directed Acyclic Graph & Linear & Dynamic approach \\ \hline
%     Interval Graph & $O(n^{4})$ & Dynamic approach \citep{inproceedings} \\ \hline
%     Circular Arc Graph & Polynomial & Dynamic approach \citep{MERTZIOS2014383} \\ \hline
%     Co-compatibility Graph & $O(n^{7})$ & Based on Hasse diagram \\ \hline 
    
%     \end{tabular}}
% \end{frame}
% \begin{frame}{Observation}
%     \begin{itemize}
%         \item The longest path problem is solvable in polynomial time 
% on any class of graphs with bounded tree-width or bounded clique-width, 
% such as the distance-hereditary graphs.
%         \item Finally, it is clearly NP-hard 
% on all graph classes on which the Hamiltonian path problem is NP-hard, 
% such as on split graphs, circle graphs, and planar graphs.
%     \end{itemize}
    
% \end{frame}

