\begin{frame}{Reduction from 3-SAT formula}
\setbeamercovered{dynamic}
\begin{block}{Sufficiency}
If $G$ has a path of length at least $k$, then $Y$ is satisfiable.
\end{block}

\begin{itemize} 
    \item<1-> Since $G$ has a path $P$ of length at least $k$, the value assigned to each literal $x_i$ will have a corresponding path($True$ or $False$) to it.
        
   
         
    \item<2-> For a literal, $x_{i}$, we assign $True$ to $x_i$ if $P$ contains the upper/true path in the loop of $x_{i}$ and assign $False$ if $P$ contains the lower path. 
     \item<3-> Due to the construction of $G$, for each literal in clause $c_j$, the complement term will have two edges with that clause, one edge with $c_{j_{in}}$ and another edge with $c_{j_{out}}$.
    
    \item<4-> If a clause $c_j$ contains the literal $x_i$, then $True$ is assigned to $x_i$ which will make $c_j$ $True$. And if a clause $c_j$ contains the literal $\overline{x_i}$, then the value assigned to $\overline{x_i}$ is  $False$ which will lead $c_j$ to being $True$.
     
     
    \item<5-> This will hold for each clause in $Y$. So $Y$ is satisfiable. 
        
\end{itemize} 
\end{frame}


% \begin{frame}{Reduction from 3-SAT formula}
% \setbeamercovered{dynamic}
% \begin{block}{Necessity}
% If Y is satisfiable, then G has a path of length of at least k.
% \end{block}

%   \begin{itemize} 
       
%         \item<1-> Say for clause $c_j$ if the term $x_i$ is true, we take edge from $c_{j_{in}}$ to 
%         $x_{i,j\_F}$  % $x_{i,j_F}$ % 
%         and from $x_{i,j\_F}$ to $c_{j_{out}}$ is taken. And for clause $c_j$, if the term $\overline{x_i}$ is true, we need to take edge from $c_{j_{in}}$ to $x_{i,j\_T}$ and from $x_{i,j\_T}$ to $c_{j_{out}}$. 
%         \item<2->
%         Thus for each clause, at least $2*q$ edges are taken.
        
%          \item<3->
%         To complete the path, edges from $c_{j_{in}}$ to $c_{j+1_{in}}$ must be taken.
%         \item<4-> Total $q-1$ edges are taken. 
%         \item<5-> Edges from $s$ to $s_1$,
% $e_n$ to $c_{start}$, $c_{start}$ to $c_{1_{in}}$ and $c_{q_{out}}$ to $t$ is taken. 
% \item<6->Thus total $n*(q+1) + n-1 + 2*q + 4$ edges is taken. So the length of the path is $k$.
        
% \end{itemize} 


% \end{frame}